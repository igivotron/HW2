\documentclass[11pt,answers]{exam}
\usepackage{graphicx}
\usepackage{packages}
\usepackage{geometry}
\usepackage{amsmath}
\usepackage{caption}
\geometry{top=2cm, bottom=2cm, left=2cm, right=2cm}


\begin{document}
\noindent
\vspace{1cm}
LMECA2550 \hfill Igor Grégoire \hfill  2025-2026\\[-2mm]
\rule{\linewidth}{0.5pt}

\begin{center}

{\large LMECA2550-HW2: Turbojet cycle analysis}
\bigskip

\end{center}
\hrule

\vspace{0.5cm}

\section*{Reminders}
We will use the nomenclature used in lecture to characterize the propulsion systems.\\
As a reminder, we define the following variables:\\

\begin{center}
    
\begin{minipage}{0.4\textwidth}
\begin{itemize}
    \item $\pi_d = \frac{p_{t2}}{p_{t0}}$
    \item $\pi_c = \frac{p_{t3}}{p_{t2}}$
    \item $\pi_b = \frac{p_{t4}}{p_{t3}}$
    \item $\pi_t = \frac{p_{t5}}{p_{t4}}$
    \item $\pi_n = \frac{p_{te}}{p_{t7}}$
    \item $\pi_{AB} = \frac{p_{t7}}{p_{t5}}$
    \item $\pi_{r} = \frac{p_{t0}}{p_{0}}$
\end{itemize}
\end{minipage}
\hspace{5pt} % <<< réduit l'espace entre les deux colonnes
\begin{minipage}{0.3\textwidth}
\begin{itemize}
    \item $\tau_d = \frac{T_{t2}}{T_{t0}}$
    \item $\tau_c = \frac{T_{t3}}{T_{t2}}$
    \item $\tau_b = \frac{T_{t4}}{T_{t3}}$
    \item $\tau_t = \frac{T_{t5}}{T_{t4}}$
    \item $\tau_n = \frac{T_{te}}{T_{t7}}$
    \item $\tau_{AB} = \frac{T_{t7}}{T_{t5}}$
    \item $\tau_{r} = \frac{T_{t0}}{T_{0}}$
    \item $\tau_{\lambda} = \frac{c_{p,t} T_{t4}}{c_{p,c}T_{t0}}$
\end{itemize}
\end{minipage}


\begin{figure}[H]
    \centering
    \includegraphics[width=0.8\linewidth]{images/jet.png}
    \caption{Turbojet}
    \label{fig:turbojet}
\end{figure}

\end{center}
\newpage

\section{Turbojet optimum compressor ratio}
We want to derive the optimum pressure and temperature ratio across the compressor of an ideal turbojet without after burner. We consider an ideal turbojet: $\tau_d = 1 $, $\pi_d$ = 1 $p_{t,e}=p_{t,5}$ $\tau_n = 1$, $\pi_n  =1$, $\pi_b = 1$, $\pi_{AB} =1$
\subsection{Specific thrust}
We start from the thrust formula and derive the specific thrust formula from it.
\begin{align*}
    F &= \dot{m}(u_e - u_0) \\
    \frac{F}{\dot{m}} &= u_e - u_0 & \text{We know that } u=Mc\\
    \frac{F}{\dot{m}} &= \left( \frac{c_e}{c_0} M_e - M_0 \right)c_0  & c = \sqrt{\gamma R T}\\
    \frac{F}{\dot{m}} &= \left( \frac{\sqrt{\gamma R T_e}}{\sqrt{\gamma R T_0}} M_e - M_0 \right) \sqrt{\gamma R T_0}\\
\end{align*}


\begin{equation}
        \frac{F}{\dot{m}} = \left( \sqrt{\frac{T_e}{T_0}} M_e - M_0 \right)c_0
        \label{eq:ST0}
\end{equation}

\subsubsection{Find $M_e$}
We take the total enthalpy and get $M_e$ from it using the engine parameters.\\

\begin{align*}
    h_{t,e} &= h_e + \frac{u_{e}^2}{2} \\
    T_{te} &= T_e + \frac{u_{e}^2}{2 c_p} \\
    \frac{T_{t,e}}{T_e} &= 1 + \frac{\gamma -1}{2} M_{e}^2\\
    M_{e}^2 &= \frac{2}{\gamma - 1} ((\frac{p_{t,e}}{p_e})^{\frac{\gamma-1}{\gamma}}-1) & \text{We that }\frac{p_{t,e}}{p_e} = \pi_r \pi_c \pi_t \\
    M_{e}^2 &= \frac{2}{\gamma - 1} ((\pi_r \pi_c \pi_t)^{\frac{\gamma-1}{\gamma}}-1) & \pi^{\frac{\gamma -1}{\gamma}} = \tau
\end{align*}

\begin{equation}
    M_{e}^2 = \frac{2}{\gamma - 1} (\tau_r \tau_c \tau_t-1)
    \label{eq:Me0}
\end{equation}

\subsubsection{Find $\frac{T_e}{T_0}$}
We want to describe $\frac{T_e}{T_0}$ as a function of engine parameters.\\

\begin{align*}
    \frac{T_{t,e}}{T_0} &= \frac{T_{t,e}}{T_0} \frac{T_{t,2}}{T_{t,0}} \frac{T_{t,3}}{T_{t,2}} \frac{T_{t,4}}{T_{t,3}} \frac{T_{t,5}}{T_{t,4}} \frac{T_{t,e}}{T_{t,5}}\\
    \frac{T_{t,e}}{T_0} &= \tau_r \tau_c \tau_b \tau_t
\end{align*}

\begin{align*}
    \frac{T_{t,e}}{T_0} &= (\frac{p_{t,e}}{p_e})^{\frac{\gamma-1}{\gamma}} = (\pi_r \pi_c \pi_t)^{\frac{\gamma-1}{\gamma}}\\
    \frac{T_{t,e}}{T_0} &= \tau_r \tau_c \tau_t
\end{align*}

$$\frac{T_e}{T_0} = \frac{T_{t,e}/T_0}{T_{t,e}/T_e}$$
\begin{equation}
    \frac{T_e}{T_0} = \tau_b
    \label{eq:TeT00}
\end{equation}
\subsubsection{Find $\tau_t$}
We use the equilibrium between the compressor and the turbine:
\begin{align*}
    \dot{W}_c &= \dot{W}_t\\
    \dot{m} C_p T_{t,4} (1-\frac{T_{t,5}}{T_{t,4}}) &= \dot{m} C_p T_{t,2} (\frac{T_{t,3}}{T_{t,2}}-1)\\
    (1 - \tau_t) &= \frac{T_{t,2}}{T_{t,4}} (\tau_c -1)
\end{align*}

\begin{equation}
    \tau_t = 1- \frac{\tau_r}{\tau_\lambda}(\tau_c -1)
    \label{eq:taut0}
\end{equation}

\subsubsection{Find $\tau_b$}

\begin{align*}
    \tau_b = \frac{T_{t,4}}{T_{t,3}} = \frac{T_{0}}{T_{t,0}} \frac{T_{t,0}}{T_{t,2}} \frac{T_{t,2}}{T_{t,3}}\frac{T_{t,4}}{T_{t,0}} = \frac{1}{\tau_b} \frac{1}{\tau_d} \frac{1}{\tau_c} \tau_\lambda
\end{align*}

\begin{equation}
   \tau_b = \frac{\tau_\lambda}{\tau_r \tau_c}
   \label{eq:taub0}
\end{equation}

\subsubsection{Back to the Specific thrust}
We inject the terms \ref{eq:Me0} \ref{eq:TeT00}  \ref{eq:taut0} \ref{eq:taub0} computed up above in the specific thrust equation \ref{eq:ST0}:

\begin{align*}
    \frac{F}{\dot{m}} &= \left( \sqrt{\frac{T_e}{T_0}} M_e - M_0 \right)c_0\\
    \frac{F}{\dot{m}} &= \left( \sqrt{\frac{\tau_\lambda}{\tau_r \tau_c} \frac{2}{\gamma - 1} (\tau_r \tau_c \tau_t-1)} - M_0 \right)c_0\\
\end{align*}

\begin{equation}
    \frac{F}{\dot{m}} = \left( \sqrt{\frac{\tau_\lambda}{\tau_r \tau_c} \frac{2}{\gamma - 1} ((1-\frac{\tau_r}{\tau_\lambda}(\tau_c -1))\tau_r \tau_c -1)} - M_0 \right)c_0
    \label{eq:ST0dev}
\end{equation}

\subsection{Specific thrust optimization}
Now that we retrieve an equation characterizing the specific thrust \ref{eq:ST0dev}, we can find $\tau_c$ and $\pi_c$ optimizing the specific thrust.\\
We first need to reshape the function \ref{eq:ST0dev} to simplify the calculation:
\begin{equation}
    (\frac{u_e}{c_0})^2 = \frac{2}{\gamma - 1}  \frac{\tau_\lambda}{\tau_r \tau_c} ((1-\frac{\tau_r}{\tau_\lambda}(\tau_c -1))\tau_r \tau_c -1)
\end{equation}

We can now easily calculate $\frac{\partial}{\partial \tau_c} (\frac{u_e}{c_0})^2 = 0$ that maximizes $\frac{F}{\dot{m}}$

\begin{align*}
    \frac{\partial}{\partial \tau_c} (\frac{2}{\gamma - 1}  \frac{\tau_\lambda}{\tau_r \tau_c} ((1-\frac{\tau_r}{\tau_\lambda}(\tau_c -1))\tau_r \tau_c -1)) &= 0\\
    -\frac{\tau_{r}^2}{\tau_\lambda}+\frac{1}{\tau_{c}^2} &= 0\\
    \tau_c &= \frac{\pm \sqrt{\tau_\lambda}}{\tau_r} & \text{We reject the negative solution}
\end{align*}

\begin{equation}
    \tau_c = \frac{\sqrt{\tau_\lambda}}{\tau_r}
\end{equation}

We know that $\pi_c$ and $\tau_c$ are linked by $\pi_c = \tau_c^{\frac{\gamma}{\gamma-1}}$ so we can find $\pi_c$

\begin{equation}
    \pi_c = ( \frac{\sqrt{\tau_\lambda}}{\tau_r})^{\frac{\gamma}{\gamma-1}}
\end{equation}

\subsection{Comparaison with Brayton cycle}

\color{red} TODO: Add comparaison with Brayton cycle \color{black}

\section{Hybrid cycle after-burning turbojet}
\subsection{Specific thrust}
We carry out the analysis for a turbojet equipped with an after-burner while introducing polytropic efficiencies for the turbine and compressor: $\eta_t$, $\eta_c$. We need to introfuce a new parameters to unsure the maximum temperature at the after-burner outlet: $\tau_{\lambda, AB} = \frac{T_{t,7}}{T_0}$\\
We derive the equation in the same fashion as for the case without after-burner. Equation \ref{eq:ST0}, \ref{eq:taut0} and \ref{eq:taub0} are not modified.

$$\frac{F}{\dot{m}} = \left( \sqrt{\frac{T_e}{T_0}} M_e - M_0 \right)c_0$$

$$\tau_t = 1- \frac{\tau_r}{\tau_\lambda}(\tau_c -1)$$
$$\tau_b = \frac{\tau_\lambda}{\tau_r \tau_c}$$

\subsubsection{Find $M_e$}
We start from find in the first part \ref{eq:Me0} and introduce the efficiencies: $\pi_{c}^{\frac{\gamma-1}{\gamma}} = \tau_{c}^{\eta_c}$ and $\pi_{t}^{\frac{\gamma-1}{\gamma}} = \tau_{t}^{1/\eta_t}$.

\begin{align*}
    M_{e}^2 &= \frac{2}{\gamma - 1} ((\pi_r \pi_c \pi_t)^{\frac{\gamma-1}{\gamma}}-1)
\end{align*}

\begin{equation}
    M_{e}^2 = \frac{2}{\gamma - 1} (\tau_r \tau_{c}^{\eta_c} \tau_{t}^{1/\eta_t}-1)
    \label{eq:Me1}
\end{equation}

\subsubsection{Find $\frac{T_e}{T_0}$}
We redo all the calculation, taking $\tau_{AB}$ into account this time.
\begin{align*}
    \frac{T_{t,e}}{T_0} &= \frac{T_{t,e}}{T_0} \frac{T_{t,2}}{T_{t,0}} \frac{T_{t,3}}{T_{t,2}} \frac{T_{t,4}}{T_{t,3}} \frac{T_{t,5}}{T_{t,4}} \frac{T_{t,7}}{T_{t,5}} \frac{T_{t,e}}{T_{t,7}}\\
    \frac{T_{t,e}}{T_0} &= \tau_r \tau_c \tau_b \tau_t \tau_{AB}
\end{align*}

\begin{align*}
    \frac{T_{t,e}}{T_0} &= (\frac{p_{t,e}}{p_e})^{\frac{\gamma-1}{\gamma}} = (\pi_r \pi_c \pi_t)^{\frac{\gamma-1}{\gamma}}\\
    \frac{T_{t,e}}{T_0} &= \tau_r \tau_{c}^{\eta_c} \tau_{t}^{1/\eta_t}
\end{align*}

$$\frac{T_e}{T_0} = \frac{T_{t,e}/T_0}{T_{t,e}/T_e}$$
\begin{equation}
    \frac{T_e}{T_0} = \tau_{c}^{1-\eta_c} \tau_{t}^{1 - 1/\eta_t} \tau_b \tau_{AB} 
    \label{eq:TeT01}
\end{equation}

\subsubsection{Find $\tau_{AB}$}

\begin{align*}
    \tau_{AB} = \frac{T_{t,7}}{T_{t,5}} &= \frac{T_{0}}{T_{t,0}}\frac{T_{t,0}}{T_{t,2}}\frac{T_{t,2}}{T_{t,3}}\frac{T_{t,3}}{T_{t,4}}\frac{T_{t,4}}{T_{t,5}}\frac{T_{t,7}}{T_{0}}\\
    \tau_{AB} &= \frac{1}{\tau_r} \frac{1}{\tau_d} \frac{1}{\tau_c} \frac{1}{\tau_b} \frac{1}{\tau_t} \tau_{\lambda, AB}
\end{align*}

\begin{equation}
   \tau_{AB} = \frac{\tau_{\lambda, AB}}{\tau_\lambda \tau_t}
   \label{eq:tauAB}
\end{equation}

\subsubsection{Back to specific thrust}
We consider that the fuel mass flow injected in the after-burner is negligible compared to the air mass flow. We inject the terms \ref{eq:Me1} \ref{eq:TeT01}  \ref{eq:taut0} \ref{eq:taub0} and \ref{eq:tauAB} computed up above in the specific thrust equation \ref{eq:ST0}:
\begin{align*}
    (\frac{u_e}{c_0})^2 &= \frac{T_e}{T_0} M_e^2 \\
    (\frac{u_e}{c_0})^2 &= \tau_{c}^{1-\eta_c} \tau_{t}^{1 - 1/\eta_t} \tau_b \tau_{AB} \frac{2}{\gamma - 1} (\tau_r \tau_{c}^{\eta_c} \tau_{t}^{1/\eta_t}-1)\\
    (\frac{u_e}{c_0})^2 &= \frac{2}{\gamma - 1} \tau_{\lambda, AB} (1 - \frac{1}{\tau_r \tau{_c}^{\eta_c} \tau{_t}^{1/\eta_t}})\\
\end{align*}

\begin{equation}
    \frac{F}{\dot{m}} = \left( \sqrt{\frac{2}{\gamma - 1} \tau_{\lambda, AB} (1 - \frac{1}{\tau_r \tau{_c}^{\eta_c} \tau{_t}^{1/\eta_t}})} - M_0 \right)c_0
    \label{ST1}
\end{equation}

\subsubsection{Specific thrust optimization}
We can now easily calculate $\frac{\partial}{\partial \tau_c} (\frac{u_e}{c_0})^2 = 0$ that maximizes $\frac{F}{\dot{m}}$.\\

\begin{align*}
    \frac{\partial}{\partial \tau_c} (\frac{2}{\gamma - 1} \tau_{\lambda, AB} (1 - \frac{1}{\tau_r \tau{_c}^{\eta_c} \tau{_t}^{1/\eta_t}})) &= 0\\
\end{align*}

\begin{equation}
    \tau_c = \frac{\eta_c \eta_t}{1+ \eta_c \eta_t} \left( \frac{\tau_\lambda}{\tau_r} + 1\right)
\end{equation}

\subsection{Ramjet specific thrust}
In the case of a ramjet, there is no compressor and turbine. We have $\tau_c = 1$ and $\tau_t = 1$. We can simplify the specific thrust equation found previously \ref{ST1}:

\begin{equation}
    \frac{F}{\dot{m}} = \left( \sqrt{\frac{2}{\gamma - 1} \tau_{\lambda, AB} (1 - \frac{1}{\tau_r}}) - M_0 \right)c_0
    \label{STram}
\end{equation}

\subsection{Transition Mach number}
We want to find the Mach number for which the specific thrust of the turbojet with after-burner equals the specific thrust of the ramjet. We equal equations \ref{ST1} and \ref{STram}:

\begin{align*}
    \tau_{\lambda, AB} (1 - \frac{1}{\tau_r \tau{_c}^{\eta_c} \tau{_t}^{1/\eta_t}}) &= \tau_{\lambda, AB} (1 - \frac{1}{\tau_r})\\
    \frac{1}{\tau_r \tau{_c}^{\eta_c} \tau{_t}^{1/\eta_t}} &= \frac{1}{\tau_r}\\
    \tau{_c}^{\eta_c} \tau{_t}^{1/\eta_t} &= 1\\
    \tau_{c} ^{\eta_c \eta_t} \tau_t &= 1\\
    \tau_{c} ^{\eta_c \eta_t} (1- \frac{\tau_r}{\tau_\lambda}(\tau_c -1)) &= 1\\
    \tau_\lambda \frac{1-\tau_{c}^{-\eta_c \eta_t}}{\tau_c -1} &= \tau_r & \text{We know that } \tau_r = 1 + \frac{\gamma -1}{2} M^2
\end{align*}

\begin{equation}
    M_{0} = \sqrt{\frac{2}{\gamma -1} (\tau_\lambda \frac{1-\tau_{c}^{-\eta_c \eta_t}}{\tau_c -1} -1)}
    \label{Mtrans}
\end{equation}

\subsubsection{Numerical application}
We consider the following parameters:
\begin{itemize}
    \item $\gamma = 1.4$
    \item $\pi_c = 20$
    \item $T_{t,4} = 1400 K$
    \item $T_{t,0} = 216 K$
    \item $\eta_c =  \eta_t = 0.9$
    \item $\tau_\lambda = \frac{T_{t,4}}{T_{t,0}} = \frac{1400}{216} = 6.48$
    \item $\tau_c = \pi_c^{\frac{\gamma -1}{\gamma} \frac{1}{\eta_c}} = 2.588$
\end{itemize}

We can now compute the transition Mach number using equation \ref{Mtrans}:\\

$$\boxed{M_{0} = 2.44}$$

\subsection{Graphical representation}
We plot the specific thrust of the turbojet with after-burner \ref{ST1} and the specific thrust of the ramjet \ref{STram} as a function of the Mach number to visualize the transition Mach number computed previously. We use the same parameters as in the numerical application part.
\begin{figure}[H]
    \centering
    \includegraphics[width=\linewidth]{images/Specific_Thrust_vs_Mach_Number.pdf}
    \caption{Specific Thrust vs Mach Number for Turbojet and Ramjet}
    \label{fig:STvsM}
\end{figure}

The graph shows that the transition Mach number is around 2.4, which is consistent with the value computed previously.
We can also observe that the ramjet specific thrust becomes higher than that of the turbojet equipped with after-burner after this Mach number, confirming the advantage of ramjets at high speeds.\\
It is interesting to note that past Mach 3.5, the turbojet doesn't produce any more thrust.

\subsection{T-S Diagram}
We plot the T-S diagram of the turbojet with after-burner at a subsonic speed and transition Mach number. We use the same parameters as in the numerical application part.

\begin{figure}
    \centering
    \includegraphics[width=\linewidth]{images/TSM.pdf}
    \caption{T-S Diagram of Turbojet and Ramjet at M=0.8 and M=2.44}
    \label{fig:TSdiagram}
\end{figure}

We can assess the different components of the engine on the T-S diagram.
First, we observe that the ram effect (0-2) is more present at M=2.44 than at M=0.8, as expected since the ram effect increases with the Mach number.\\
We can also observe at that at M=2.44, the compressor generates large amount of heat and need a lot of work from the turbine to compress the air.\\
At lower Mach number (M=0.8), the temperature increases a lot in the combustion chamber, as well as the entropy while at M=2.44, the temperature and entropy increase are more moderate. This is due to the fact that at higher Mach number, the air entering the combustor is already at a higher temperature due to the ram effect and the compressor.\\
We can see that the turbine need to produce a huge amount of work at M=2.44 to drive the compressor compared to M=0.8. At the turbine outlet, the air is at lower temperature and much lower entropy at M=2.44 compared with M=0.8\\
The after-burner increases significantly the temperature and entropy of the air before the nozzle at both Mach numbers, providing additional thrust with huge fuel consumption.\\
On the other and, the ramjet cycle shows a more straightforward process with a continuous increase in temperature and entropy from inlet to nozzle. We can see the ram effect (0-2) followed by the combustion process (2-5). Pratically all the combustion process is "post combustion" as there is no turbine or compressor but we divide the process in two part is the same fashion as the turbojet.\\
\subsection{Specific fuel thrust consumption}
We want to analyse the specific fuel thrust consumption (TSFC) of the turbojet with after-burner. We first need to define the TSFC:
\begin{equation}
    TSFC = \frac{\dot{m}_{f,tot}}{F} = \frac{f_{tot}}{F/\dot{m}}
\end{equation}

We also need to define the fuel-to-air ratio $f_{tot}$:\\
\subsubsection{Turbojet fuel-to-air ratio}
For the burner we have:
\begin{align*}
    \dot{m}_0 h_{t,3} + \dot{m}_f LHV &= (\dot{m}_0 + \dot{m}_f) h_{t,4} & \text{We assume that } \dot{m}_0 + \dot{m}_f = \approx \dot{m}_0 \\
    \dot{m}_0 c_p T_{t,3} + \dot{m}_f LHV &= \dot{m}_0 c_p T_{t,4} & f = \frac{\dot{m}_f}{\dot{m}_0}\\
    f = \frac{c_p (T_{t,4} - T_{t,3})}{LHV}\\
    f = \frac{c_p T_{0} (\tau_\lambda - \tau_r \tau_c)}{LHV}
\end{align*}
For the after-burner we have:
\begin{align*}
    \dot{m}_0 h_{t,5} + \dot{m}_{f,AB} LHV &= (\dot{m}_0 + \dot{m}_{f,AB}) h_{t,7} & \text{We assume that } \dot{m}_0 + \dot{m}_{f,AB} = \approx \dot{m}_0 \\
    \dot{m}_0 c_p T_{t,5} + \dot{m}_{f,AB} LHV &= \dot{m}_0 c_p T_{t,7} & f_{AB} = \frac{\dot{m}_{f,AB}}{\dot{m}_0}\\
    f_{AB} &= \frac{c_p (T_{t,7} - T_{t,5})}{LHV} & \frac{T_{t,5}}{T_0} = \frac{T_{t,0}}{T_0} \frac{T_{t,2}}{T_{t,0}} \frac{T_{t,3}}{T_{t,2}} \frac{T_{t,4}}{T_{t,3}} \frac{T_{t,5}}{T_{t,4}}\\
    f_{AB} &= \frac{c_p T_{0} (\tau_{\lambda, AB} - \tau_r \tau_c \tau_b \tau_t)}{LHV}
\end{align*}

The total fuel-to-air ratio is:
\begin{equation}
    f_{tot} = f + f_{AB} = \frac{c_p T_{0} (\tau_\lambda - \tau_r \tau_c)}{LHV} + \frac{c_p T_{0} (\tau_{\lambda, AB} - \tau_r \tau_c \tau_b \tau_t)}{LHV}
\end{equation}

\subsubsection{Ramjet fuel-to-air ratio}
For the ramjet we have:
\begin{align*}
    \dot{m}_0 h_{t,2} + \dot{m}_f LHV &= (\dot{m}_0 + \dot{m}_f) h_{t,7} & \text{We assume that } \dot{m}_0 + \dot{m}_f = \approx \dot{m}_0 \\
    \dot{m}_0 c_p T_{t,2} + \dot{m}_f LHV &= \dot{m}_0 c_p T_{t,7} & f = \frac{\dot{m}_f}{\dot{m}_0}\\
    f = \frac{c_p (T_{t,7} - T_{t,2})}{LHV}\\
    f = \frac{c_p T_{0} (\tau_{\lambda, AB} - \tau_r)}{LHV}
\end{align*}

\subsubsection{Numerical application}
We have plotted the TSFC of the turbojet with after-burner and the ramjet as a function of the Mach number using the same parameters as in the previous part.
\begin{figure}[H]
    \centering
    \includegraphics[width=\linewidth]{images/TSFC.pdf}
    \caption{TSFC vs Mach Number for Turbojet and Ramjet}
    \label{fig:TSFCvsM}
\end{figure}

We observe that the TSFC of the ramjet is lower than that of the turbojet with after-burner for Mach numbers higher than the transition Mach number (M=2.44). This confirms the advantage of ramjets at high speeds.\\
We can also observe that the TFSC of the turbojet with after-burner increases significantly with the Mach number, indicating that the fuel consumption becomes less efficient at higher speeds. On the other hand, the TSFC of the ramjet decreases then remains relatively constant across the high Mach number range.\\

\section{Off-design analysis of turbojet}

\end{document}

